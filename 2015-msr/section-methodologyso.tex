
\section{Mining StackOverflow} \label{sec:methodologyso}

In this section we describe our methodology to retrieve the data we used for our analysis.
We first begin describing how we mined source code repositories and then how we analysed the Stackoverflow database.

The complete scripts and results used for our study are available online~\footnote{\url{https://bitbucket.org/acuarica/java-unsafe-analysis}}.

\subsection{Source Code repositories}

For source code mining we searched in Boa \cite{Dyer-Nguyen-Rajan-Nguyen-13}. 
The BOA infrastructure allows the user to navigate the parsed AST of source code.

Our Boa script looks for \texttt{sun.misc.Unsafe} as either an import or fully qualified name where a type may appear.
In case that we found a use of \smu{} we proceed to determine which method is used.


It is possible to group methods in \smu{} by functionality.
Table \ref{table:groups} shows all methods (without overloads) grouped by functionality.


\begin{table}[ht]
\centering
\caption{Functional groups of \smu{}}
\label{table:groups}
\begin{tabular}{|r|l|l|} \hline
Group  & Methods \\ \hline \hline
Array  & arrayBaseOffset arrayIndexScale \\ \hline
CAS    & compareAndSwapInt compareAndSwapLong \\
       & compareAndSwapObject \\ \hline
Class  & defineAnonymousClass defineClass ensureClassInitialized \\ \hline
Get    & getBoolean getByte getChar getDouble getFloat getInt \\
       & getIntVolatile getLoadAverage getLong getLongVolatile \\
       & getObject getObjectVolatile getShort getBooleanVolatile \\
       & getDoubleVolatile getFloatVolatile getByteVolatile \\
       & getCharVolatile getShortVolatile \\ \hline
Memory & addressSize allocateMemory copyMemory freeMemory \\
       & getAddress pageSize putAddress \\ 
       & reallocateMemory setMemory \\ \hline
Offset & fieldOffset objectFieldOffset staticFieldBase staticFieldOffset \\ \hline
Park   & park unpark \\ \hline
Put    & putBoolean putByte putChar putDouble putFloat putInt \\
       & putIntVolatile putLong putLongVolatile putObject \\
       & putObjectVolatile putOrderedInt putOrderedLong \\
       & putOrderedObject putShort putCharVolatile \\ 
       & putOrderedInt putBooleanVolatile putShortVolatile \\ 
       & putFloatVolatile putByteVolatile putDoubleVolatile \\ \hline
Single & allocateInstance throwException \\ \hline
Monitor & monitorEnter monitorExit tryMonitorEnter \\ \hline

\end{tabular}
\end{table}

% Not used methods
% monitorEnter monitorExit tryMonitorEnter
% defineAnonymousClass
% shouldBeInitialized ???

% getBooleanVolatile getDoubleVolatile getFloatVolatile getByteVolatile getCharVolatile getShortVolatile

% putOrderedInt putBooleanVolatile putShortVolatile putFloatVolatile putByteVolatile putDoubleVolatile putCharVolatile


\subsubsection*{Reflection}

%\begin{verbatim}
%Class<?> unsafeClass = Class.forName("sun.misc.Unsafe");
%Field unsafeField = unsafeClass.getDeclaredField("theUnsafe");
%unsafeField.setAccessible(true);
%Object unsafe = unsafeField.get(null);
%int addressSize = ((Number) %unsafeClass.getMethod("addressSize").invoke(unsafe)).intValue();
%\end{verbatim}

What happens with other uses such as reflection? It is not detected but it uses Unsafe. There should be a way to measure this kind of use.

Look for problematic uses of the API, and some use patterns.


\subsection{Stackoverflow}

Google search for sun.misc.unsafe site:stackoverflow.com
returns about 1,360 results.
