
\section{Mining Boa infrastructure} \label{sec:methodologyboa}

In this section we describe our methodology to mine the data we used for our analysis.
We first begin describing how we mined source code repositories. The next section describes how we analysed the Stackoverflow database.

The complete scripts and results used for our study are available online~\footnote{\url{https://bitbucket.org/acuarica/java-unsafe-analysis}}.

\subsection{Static usage}

For source code mining we searched in Boa \cite{Dyer-Nguyen-Rajan-Nguyen-13}. 
The BOA infrastructure allows the user to navigate the parsed AST of source code.
Notice that BOA does not provide type resolution analysis.
Therefore we develop some heuristics to determine where and how \smu{} is used.

The following assumptions hold:
\begin{itemize}
\item \smu{} class is \texttt{final} 
\item inherits directly from \texttt{java.lang.Object}
\item Its public methods (except for \texttt{getUnsafe}) are instance methods
\end{itemize}

Thus, our Boa script~\footnote{\url{https://bitbucket.org/acuarica/java-unsafe-analysis/raw/master/unsafe-analysis/unsafe.boa}} looks for \texttt{sun.misc.Unsafe} as either an import or fully qualified name where a type may appear.
In case that we found a use of \smu{} we proceed to determine which method is used.
If \smu{} is found in a compilation unit, then all call sites in the compilation unit are analysed.
If a call site target is one of the \smu{} methods, then we can conclude that \smu{} is used and that method is called.

\subsection{Reflection usage}

What happens if \smu{} is used through reflection?
Sometimes this API is used through reflection to avoid the compilation dependency (remember that \smu{} is not part of the public API).

For instance, listing~\ref{lst:reflection} shows how to take the address size of the host machine.
First it obtains an instance of \smu{} by reflection.
Then, again by reflection, invoke the method \texttt{addressSize}, which returns the address size of the host machine.


%\definecolor{codegreen}{rgb}{0,0.6,0}
%\definecolor{codegray}{rgb}{0.5,0.5,0.5}
%\definecolor{codepurple}{rgb}{0.58,0,0.82}
%\definecolor{backcolour}{rgb}{0.95,0.95,0.92}
 
\lstdefinestyle{java}{
    language=Java,
 %   backgroundcolor=\color{backcolour},   
%    commentstyle=\color{codegreen},
%    keywordstyle=\color{magenta},
  %  numberstyle=\tiny\color{codegray},
%    stringstyle=\color{codepurple},
    basicstyle=\footnotesize,
    breakatwhitespace=false,         
    breaklines=true,                 
    captionpos=b,                    
    keepspaces=true,                 
%    numbers=left,                    
%    numbersep=5pt,                  
    showspaces=false,                
    showstringspaces=false,
    showtabs=false,                  
    tabsize=2,
    frame=shadowbox
}

\lstset{style=java}
\lstset{escapechar=§}

\begin{lstlisting}[caption=Use of sun.misc.Unsafe with reflection,label=lst:reflection]
Class<?> cls = Class.forName("sun.misc.Unsafe");
Field theUnsafe = cls.getDeclaredField("theUnsafe");
theUnsafe.setAccessible(true);
Object unsafe = theUnsafe.get(null);
Method method = cls.getMethod("addressSize");
Object value = method.invoke(unsafe);
int addressSize = ((Number) value).intValue();
\end{lstlisting}

We are aware of this situation and also look for the string literal \texttt{"sun.misc.Unsafe"} to see when it is used through reflection.

\subsection{Avoiding duplicates}

Usually a code repository has branches, to allow the team to introduce new features without interrupt the main source of development.
In Boa, all \smu{} uses are reported, regardless if it is the same file in different branches.
Sometimes it happens that the same file, with the same uses, has different package in different branches.
Therefore we need to take into account this situation.
To determine if a file is a duplicate in another branch, we look for the same file name and uses, if we found the same file and the same uses, we assume that the file is in a branch and we exclude it.

\subsection{OpenJDK forks}


\subsection{Metadata}

Also we are interested in some metadata information.
For each project that uses \smu{}, either statically or dinamically, our Boa script also retrieves :
\begin{itemize}
\item the size of the project (in number of AST nodes)
\item the number of revisions in the repository.
\item the lifetime of the repository (time from the first to the last commit)
\end{itemize}

It is possible to group methods in \smu{} by functionality.
Table \ref{table:groups} shows all methods (without overloads) grouped by functionality.


\begin{table}[ht]
\centering
\begin{tabular}{|r|l|l|} \hline
Group  & Methods \\ \hline \hline
Array  & arrayBaseOffset arrayIndexScale \\ \hline
CAS    & compareAndSwapInt compareAndSwapLong \\
       & compareAndSwapObject \\ \hline
Class  & defineAnonymousClass defineClass ensureClassInitialized \\ \hline
Get    & getBoolean getByte getChar getDouble getFloat getInt \\
       & getIntVolatile getLoadAverage getLong getLongVolatile \\
       & getObject getObjectVolatile getShort getBooleanVolatile \\
       & getDoubleVolatile getFloatVolatile getByteVolatile \\
       & getCharVolatile getShortVolatile \\ \hline
Memory & addressSize allocateMemory copyMemory freeMemory \\
       & getAddress pageSize putAddress \\ 
       & reallocateMemory setMemory \\ \hline
Offset & fieldOffset objectFieldOffset staticFieldBase staticFieldOffset \\ \hline
Park   & park unpark \\ \hline
Put    & putBoolean putByte putChar putDouble putFloat putInt \\
       & putIntVolatile putLong putLongVolatile putObject \\
       & putObjectVolatile putOrderedInt putOrderedLong \\
       & putOrderedObject putShort putCharVolatile \\ 
       & putOrderedInt putBooleanVolatile putShortVolatile \\ 
       & putFloatVolatile putByteVolatile putDoubleVolatile \\ \hline
Single & allocateInstance throwException \\ \hline
Monitor & monitorEnter monitorExit tryMonitorEnter \\ \hline

\end{tabular}
\caption{Functional groups of \smu{}}
\label{table:groups}
\end{table}

% Not used methods
% monitorEnter monitorExit tryMonitorEnter
% defineAnonymousClass
% shouldBeInitialized ???

% getBooleanVolatile getDoubleVolatile getFloatVolatile getByteVolatile getCharVolatile getShortVolatile

% putOrderedInt putBooleanVolatile putShortVolatile putFloatVolatile putByteVolatile putDoubleVolatile putCharVolatile

