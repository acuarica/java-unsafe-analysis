
\section{Introduction} \label{sec:introduction}

\smu{} is an undocumented~\footnote{\url{http://www.oracle.com/technetwork/java/faq-sun-packages-142232.html}} class provided by Oracle.
It allows the developer to access low-level programming features.
It is the equivalent to \texttt{unsafe}~\footnote{\url{http://msdn.microsoft.com/en-us/en-en/library/chfa2zb8(v=vs.90).aspx}} in C\#.

There is a trend in the late years to use \smu{}.
The main reason to use \smu{} is \emph{performance}.
Because \smu{} provides groups of methods to allow the programmer to access otherwise impossible low-level access. 


\begin{table}[ht]
\centering
\begin{tabular}{|r|l|l|} \hline
Group  & Methods \\ \hline \hline
Array  & arrayBaseOffset arrayIndexScale \\ \hline
CAS    & compareAndSwapInt compareAndSwapLong \\
       & compareAndSwapObject \\ \hline
Class  & defineAnonymousClass defineClass ensureClassInitialized \\ \hline
Get    & getBoolean getByte getChar getDouble getFloat getInt \\
       & getIntVolatile getLoadAverage getLong getLongVolatile \\
       & getObject getObjectVolatile getShort getBooleanVolatile \\
       & getDoubleVolatile getFloatVolatile getByteVolatile \\
       & getCharVolatile getShortVolatile \\ \hline
Memory & addressSize allocateMemory copyMemory freeMemory \\
       & getAddress pageSize putAddress \\ 
       & reallocateMemory setMemory \\ \hline
Offset & fieldOffset objectFieldOffset staticFieldBase staticFieldOffset \\ \hline
Park   & park unpark \\ \hline
Put    & putBoolean putByte putChar putDouble putFloat putInt \\
       & putIntVolatile putLong putLongVolatile putObject \\
       & putObjectVolatile putOrderedInt putOrderedLong \\
       & putOrderedObject putShort putCharVolatile \\ 
       & putOrderedInt putBooleanVolatile putShortVolatile \\ 
       & putFloatVolatile putByteVolatile putDoubleVolatile \\ \hline
Single & allocateInstance throwException \\ \hline
Monitor & monitorEnter monitorExit tryMonitorEnter \\ \hline

\end{tabular}
\caption{Functional groups of \smu{}}
\label{table:groups}
\end{table}

% Not used methods
% monitorEnter monitorExit tryMonitorEnter
% defineAnonymousClass
% shouldBeInitialized ???

% getBooleanVolatile getDoubleVolatile getFloatVolatile getByteVolatile getCharVolatile getShortVolatile

% putOrderedInt putBooleanVolatile putShortVolatile putFloatVolatile putByteVolatile putDoubleVolatile putCharVolatile



Our research goal is to find 

Trend using unsafe methods. But what for? Certainly you can do everithing without Unsafe API, so why using it? The key is performance.

We searched for Boa \cite{Dyer-Nguyen-Rajan-Nguyen-13}

 discouraged. 
 


\subsection{Subsection Heading Here}
Subsection text here.


\subsubsection{Subsubsection Heading Here}

%\verbatiminput{../boa-client/unsafe.boa}


This demo file is intended to serve as a ``starter file''
for IEEE conference papers produced under \LaTeX\ using
IEEEtran.cls version 1.7 and later.
% You must have at least 2 lines in the paragraph with the drop letter
% (should never be an issue)


Stackoverflow mining.

Measure error with boa.

\subsection{Reflection}

%\begin{verbatim}
%Class<?> unsafeClass = Class.forName("sun.misc.Unsafe");
%Field unsafeField = unsafeClass.getDeclaredField("theUnsafe");
%unsafeField.setAccessible(true);
%Object unsafe = unsafeField.get(null);
%int addressSize = ((Number) %unsafeClass.getMethod("addressSize").invoke(unsafe)).intValue();
%\end{verbatim}

What happens with other uses such as reflection? It is not detected but it uses Unsafe. There should be a way to measure this kind of use.

Mine github, maybe parsing html.

Look for problematic uses of the API, and some use patterns.

Bugreport stackoverflow posts.

The rest of this paper is organized as follows:
Section \ref{sec:relatedwork} presents related work.
In Section \ref{fig:} we show how to use the JNIF API.
Section 4 describes the design of JNIF.
Section 5 explains how we validated JNIF.
Section 6 evaluates JNIF’s performance against the mainstream bytecode manipulator, ASM.
Section 7 discusses limitations, and Section \ref{sec:conclusions} concludes.
