
\section{Introduction} \label{sec:introduction}

\smu{} is an undocumented~\footnote{\url{http://www.oracle.com/technetwork/java/faq-sun-packages-142232.html}} class provided by Oracle.
It allows the developer to access low-level programming features.
It is the equivalent to \texttt{unsafe}~\footnote{\url{http://msdn.microsoft.com/en-us/en-en/library/chfa2zb8(v=vs.90).aspx}} in C\#.

There is a trend in the late years to use \smu{}.
The main reason to use \smu{} is \emph{performance}.
Because \smu{} provides methods to allow the programmer to access otherwise impossible low-level details.
For instance, \smu{} contains methods to do CAS operations, the base ground to develop lock-free data structures.

Why is it important to study \smu{} API usage patterns? Because with \smu{} bad things can happen.
Sandoz~\cite{psandoz14} describe several usage patterns of \smu{}. Let's show of these examples.

Sandoz~\cite{psandoz14} did a survey to study how Unsafe is used~\footnote{\url{http://www.infoq.com/news/2014/02/Unsafe-Survey}}.
We took one step further and analyse how \smu{} is used in BOA. Moreover, we want to study how effectively people is discussing with respect to \smu{}, to that end, we analyse the Stackoverflow database.

Our research goal is to find 
Similar \cite{Dyer-Rajan-Nguyen-Nguyen-14}

Trend using unsafe methods. But what for? Certainly you can do everithing without Unsafe API, so why using it? 

 
Stackoverflow mining.

Measure error with boa.


Bugreport stackoverflow posts.


Overall, the main contributions of this paper are two-fold:
\begin{itemize}
\item We present a detailed study of how the Java \smu{} API is used and
\item We constrast this information on why this API is used based on responses from Stackoverflow.
\end{itemize}

The rest of this paper is organized as follows:
Section \ref{sec:relatedwork} presents related work.
Section \ref{sec:methodology} explains the methodology and technologies used to get our results.
Section \ref{sec:results} shows the results we obtained and Section \ref{sec:conclusions} concludes.
