
\section{Introduction} \label{sec:introduction}

The Java Virtual Machine (JVM) executes Java bytecode and provides other
services for programs written in Java, Scala, Clojure, and many other
languages. Although the JVM was designed to be portable, ``write once run
anywhere'', many JVM implementations in wide use expose an API to allow access
to low-level, non-portable features of the JVM.
This API is provided through an
undocumented~\footnote{\url{http://www.oracle.com/technetwork/java/faq-sun-packages-142232.html}}
class, \smu{}, in the Java reference implementation produced by Oracle.
The class allows the developer to access low-level programming features.

Other virtual machines provide similar functionality. For instance the
language C\# provides an \texttt{unsafe} construct on the .NET
platform.\footnote{\url{http://msdn.microsoft.com/en-us/en-en/library/chfa2zb8(v=vs.90).aspx}}

Use of \smu{} has increased recently [FIXME citation].
The main reason for this trend \smu{} is \emph{performance}.
\smu{} provides methods to allow the programmer to access low-level 
details of the virtual machine and underlying hardware that
would otherwise be impossible.
For instance, \smu{} contains methods to do compare-and-swap (CAS) operations,
needed to implement lock-free data structures.

The operations \smu{} provides can be dangerous. If misused they
can cause 
performance problems, resource leaks, deadlock, data corruption,
and even VM crashes. 

Because of the danger of using \smu{}, we aim to study how the API is used
in practice, with the longer term goal of providing safer
alternatives to \smu{} on the JVM and on other virtual machines in general.

Sandoz~\cite{psandoz14} describe several usage patterns of \smu{}. Let's show of these examples.

Sandoz~\cite{psandoz14} did a survey to study how Unsafe is used~\footnote{\url{http://www.infoq.com/news/2014/02/Unsafe-Survey}}.

In this paper, we go beyond that survey and look at 
how \smu{} is used in the wild. We systematically examine projects open source
projects from the SourceForge repository
using BOA~\cite{Dyer-Nguyen-Rajan-Nguyen-13} and analyze how these
projects use \smu{}.
Moreover, we study problems encountered using \smu{} by analyzing the
StackOverflow question/answer database.

Trend using unsafe methods. But what for? Certainly you can do everithing without Unsafe API, so why using it? 

 
Stackoverflow mining.

Measure error with boa.


Bugreport stackoverflow posts.


Overall, the main contributions of this paper are two-fold:
\begin{itemize}
\item We present a detailed study of how the Java \smu{} API is used and
\item We constrast this information on why this API is used based on responses from Stackoverflow.
\end{itemize}

The rest of this paper is organized as follows:
Section \ref{sec:relatedwork} presents related work.
Section \ref{sec:methodology} explains the methodology and technologies used to get our results.
Section \ref{sec:results} shows the results we obtained and Section \ref{sec:conclusions} concludes.
