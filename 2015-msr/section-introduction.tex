
\section{Introduction} \label{sec:introduction}

\smu{} is an undocumented~\footnote{\url{http://www.oracle.com/technetwork/java/faq-sun-packages-142232.html}} class provided by Oracle.
It allows the developer to access low-level programming features.
It is the equivalent to \texttt{unsafe}~\footnote{\url{http://msdn.microsoft.com/en-us/en-en/library/chfa2zb8(v=vs.90).aspx}} in C\#.

There is a trend in the late years to use \smu{}.
The main reason to use \smu{} is \emph{performance}.
Because \smu{} provides methods to allow the programmer to access otherwise impossible low-level details.
For instance, \smu{} contains methods to do CAS operations, the base ground to develop lock-free data structures.










































































\footnote{\url{http://cr.openjdk.java.net/~psandoz/dv14-uk-paul-sandoz-unsafe-the-situation.pdf}}

Our research goal is to find 
Similar \cite{Dyer-Rajan-Nguyen-Nguyen-14}



It is possible to group methods in \smu{} by functionality.
Table \ref{table:groups} shows all methods (without overloads) grouped by functionality.


\begin{table}[ht]
\centering
\begin{tabular}{|r|l|l|} \hline
Group  & Methods \\ \hline \hline
Array  & arrayBaseOffset arrayIndexScale \\ \hline
CAS    & compareAndSwapInt compareAndSwapLong \\
       & compareAndSwapObject \\ \hline
Class  & defineAnonymousClass defineClass ensureClassInitialized \\ \hline
Get    & getBoolean getByte getChar getDouble getFloat getInt \\
       & getIntVolatile getLoadAverage getLong getLongVolatile \\
       & getObject getObjectVolatile getShort getBooleanVolatile \\
       & getDoubleVolatile getFloatVolatile getByteVolatile \\
       & getCharVolatile getShortVolatile \\ \hline
Memory & addressSize allocateMemory copyMemory freeMemory \\
       & getAddress pageSize putAddress \\ 
       & reallocateMemory setMemory \\ \hline
Offset & fieldOffset objectFieldOffset staticFieldBase staticFieldOffset \\ \hline
Park   & park unpark \\ \hline
Put    & putBoolean putByte putChar putDouble putFloat putInt \\
       & putIntVolatile putLong putLongVolatile putObject \\
       & putObjectVolatile putOrderedInt putOrderedLong \\
       & putOrderedObject putShort putCharVolatile \\ 
       & putOrderedInt putBooleanVolatile putShortVolatile \\ 
       & putFloatVolatile putByteVolatile putDoubleVolatile \\ \hline
Single & allocateInstance throwException \\ \hline
Monitor & monitorEnter monitorExit tryMonitorEnter \\ \hline

\end{tabular}
\caption{Functional groups of \smu{}}
\label{table:groups}
\end{table}

% Not used methods
% monitorEnter monitorExit tryMonitorEnter
% defineAnonymousClass
% shouldBeInitialized ???

% getBooleanVolatile getDoubleVolatile getFloatVolatile getByteVolatile getCharVolatile getShortVolatile

% putOrderedInt putBooleanVolatile putShortVolatile putFloatVolatile putByteVolatile putDoubleVolatile putCharVolatile



Trend using unsafe methods. But what for? Certainly you can do everithing without Unsafe API, so why using it? 

 

Stackoverflow mining.

Measure error with boa.


Bugreport stackoverflow posts.


Overall, the main contributions of this paper are two-fold:
\begin{itemize}
\item We present a detailed study of how the Java \smu{} API is used and
\item We constrast this information on why this API is used based on responses from Stackoverflow.
\end{itemize}

The rest of this paper is organized as follows:
Section \ref{sec:relatedwork} presents related work.
Section \ref{sec:methodology} explains the methodology and technologies used to get our results.
Section \ref{sec:results} shows the results we obtained and Section \ref{sec:conclusions} concludes.
